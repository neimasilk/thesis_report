referensi :

load gambar

\begin{figure}
	\centering
	\includegraphics[width=0.74\textwidth]
		{pics/creative_common.png}
	\caption{\license}
	\label{fig:lisensi}
\end{figure}


buat kotak tabular :

Berdasarkan \cite{latex.intro}: \\ 
\begin{tabular}{| p{13cm} |}
	\hline 
	\\
	LaTeX is a family of programs designed to produce publication-quality 
	typeset documents. It is particularly strong when working with 
	mathematical symbols. \\	
	The history of LaTeX begins with a program called TEX. In 1978, a 
	computer scientist by the name of Donald Knuth grew frustrated with the 
	mistakes that his publishers made in typesetting his work. He decided 
	to create a typesetting program that everyone could easily use to 
	typeset documents, particularly those that include formulae, and made 
	it freely available. The result is TEX. \\	
	Knuth's product is an immensely powerful program, but one that does 
	focus very much on small details. A mathematician and computer 
	scientist by the name of Leslie Lamport wrote a variant of TEX called 
	LaTeX that focuses on document structure rather than such details. \\
	\\
	\hline
\end{tabular}

\vspace*{0.8cm}


persamaan :

%-----------------------------------------------------------------------------%
\section{Satu Persamaan}
%-----------------------------------------------------------------------------%

\noindent \begin{align}\label{eq:garis}
	\cfrac{y - y_{1}}{y_{2} - y_{1}} = 
	\cfrac{x - x_{1}}{x_{2} - x_{1}}
\end{align}

\equ~\ref{eq:garis} diatas adalah persamaan garis. 
\equ~\ref{eq:garis} dan \ref{eq:bola} sama-sama dibuat dengan perintah \bslash
align. 
Perintah ini juga dapat digunakan untuk menulis lebih dari satu persamaan. 

\noindent \begin{align}\label{eq:bola}
	\underbrace{|\overline{ab}|}_{\text{pada bola $|\overline{ab}| = r$}} 
		= \sqrt[2]{(x_{b} - x_{a})^{2} + (y_{b} - y_{a})^{2} + 
				\vert\vert(z_{b} - z_{a})^{2}}
\end{align}

%-----------------------------------------------------------------------------%
\section{Lebih dari Satu Persamaan}
\label{sec:multiEqu}
%-----------------------------------------------------------------------------%
\noindent \begin{align}\label{eq:matriks}	
	|\overline{a} * \overline{b}| &= |\overline{a}| |\overline{b}| \sin\theta 
		\\[0.2cm]
	\overline{a} * \overline{b} &=  
		\begin{array}{| c c c |}
			\hat{i} & x_{1} & x_{2} \\
			\hat{j} & y_{1} & y_{2} \\
			\hat{k} & z_{1} & z_{2} \\
		\end{array} \nonumber \\[0.2cm]
	&= \hat{i} \,
		\begin{array}{ | c c | }
			y_{1} & y_{2} \\
			z_{1} & z_{2} \\
		\end{array} 
	   + \hat{j} \,
		\begin{array}{ | c c | }
			z_{1} & z_{2} \\
			x_{1} & x_{2} \\
		\end{array} 
	   + \hat{k} \,	
		\begin{array}{ | c c | }
			x_{1} & x_{2} \\
			y_{1} & y_{2} \\
		\end{array}
		\nonumber
\end{align}

Pada \equ~\ref{eq:matriks} dapat dilihat beberapa baris menjadi satu bagian 
dari \equ~\ref{eq:matriks}. 
Sedangkan dibawah ini dapat dilihat bahwa dengan cara yang sama, \equ~
\ref{eq:gabungan1}, \ref{eq:gabungan2}, dan \ref{eq:gabungan3} memiliki nomor 
persamaannya masing-masing. 

\noindent \begin{align}\label{eq:gabungan1}	
	\int_{a}^{b} f(x)\, dx + \int_{b}^{c} f(x) \, dx = \int_{a}^{c} f(x) \, dx
		\\\label{eq:gabungan2}
	\lim_{x \to \infty} \frac{f(x)}{g(x)} = 0 \hspace{1cm} 
		\text{jika pangkat $f(x)$ $<$ pangkat $g(x)$} \\\label{eq:gabungan3}
	a^{m^{a \, ^{n}\log b }} = b^{\frac{m}{n}}
\end{align}





%-----------------------------------------------------------------------------%
\section{Membuat Tabel}
%-----------------------------------------------------------------------------%
Seperti pada gambar, tabel juga dapat diberi label dan caption. 
Caption pada tabel terletak pada bagian atas tabel. 
Contoh tabel sederhana dapat dilihat pada \tab~\ref{tab:tab1}.

\begin{table}
	\centering
	\caption{Contoh Tabel}
	\label{tab:tab1}
	\begin{tabular}{| l | c r |}
		\hline
		& kol 1 & kol 2 \\ 
		\hline
		baris 1 & 1 & 2 \\
		baris 2 & 3 & 4 \\
		baris 3 & 5 & 6 \\
		jumlah  & 9 & 12 \\
		\hline
	\end{tabular}
\end{table}

Ada jenis tabel lain yang dapat dibuat dengan \latex~berikut 
beberapa diantaranya. 
Contoh-contoh ini bersumber dari 
\url{http://en.wikibooks.org/wiki/LaTeX/Tables}

\begin{table}
	\centering
	\caption{An Example of Rows Spanning Multiple Columns}
	\label{row.spanning}
	\begin{tabular}{|l|l|*{6}{c|}}
  		\hline % create horizontal line
  		No & Name & \multicolumn{3}{|c|}{Week 1} & \multicolumn{3}{|c|}{Week 2} \\
  		\cline{3-8} % create line from 3rd column till 8th column
  		& & A & B & C & A & B & C\\
  		\hline
  		1 & Lala & 1 & 2 & 3 & 4 & 5 & 6\\
  		2 & Lili & 1 & 2 & 3 & 4 & 5 & 6\\
  		3 & Lulu & 1 & 2 & 3 & 4 & 5 & 6\\
  		\hline
	\end{tabular}
\end{table}

\begin{table}
	\centering
	\caption{An Example of Columns Spanning Multiple Rows}
	\label{column.spanning}
	\begin{tabular}{|l|c|l|}
		\hline
		Percobaan & Iterasi & Waktu \\
		\hline
		Pertama & 1 & 0.1 sec \\ \hline
		\multirow{2}{*}{Kedua} & 1 & 0.1 sec \\
 		& 3 & 0.15 sec \\ 
 		\hline
		\multirow{3}{*}{Ketiga} & 1 & 0.09 sec \\
 		& 2 & 0.16 sec \\
 		& 3 & 0.21 sec \\ 
 		\hline
	\end{tabular}
\end{table}

\begin{table}
	\centering
	\caption{An Example of Spanning in Both Directions Simultaneously}
	\label{mix.spanning}
	\begin{tabular}{cc|c|c|c|c|}
		\cline{3-6}
		& & \multicolumn{4}{|c|}{Title} \\ \cline{3-6}
		& & A & B & C & D \\ \hline
		\multicolumn{1}{|c|}{\multirow{2}{*}{Type}} &
		\multicolumn{1}{|c|}{X} & 1 & 2 & 3 & 4\\ \cline{2-6}
		\multicolumn{1}{|c|}{}                        &
		\multicolumn{1}{|c|}{Y} & 0.5 & 1.0 & 1.5 & 2.0\\ \cline{1-6}
		\multicolumn{1}{|c|}{\multirow{2}{*}{Resource}} &
		\multicolumn{1}{|c|}{I} & 10 & 20 & 30 & 40\\ \cline{2-6}
		\multicolumn{1}{|c|}{}                        &
		\multicolumn{1}{|c|}{J} & 5 & 10 & 15 & 20\\ \cline{1-6}
	\end{tabular}
\end{table}

