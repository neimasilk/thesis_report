%
% Halaman Abstract
%
% @author  Andreas Febrian
% @version 1.00
%

\chapter*{ABSTRACT}

\vspace*{0.2cm}

\noindent \begin{tabular}{l l p{11.0cm}}
	Name&: & \penulis \\
	Program&: & \program \\
	Title&: & \judulInggris \\
\end{tabular} \\ 

\vspace*{0.5cm}

\noindent 
Microarray technology has made possible the profiling of gene expressions of the entire genome in a single hybridization experiment. Since microarray data acquire tens of thousands of gene expression values simultaneously. However, the number of sample usually small. Feature selection and clustering algorithm for microarray data analysis is useful to extract cluster structure and to reduce the high dimensional microarray data and reconstruct to lower dimensional with minimum error possible. Deep learning and clustering is a machine learning method. In this research we will investigate the effectiveness of clustering after or prior dimensionality reduction. The most common deep learning architecture used for dimensionality reduction is deep believe network (DBN) and stacked auto encoder (SAE). Pre training unsupervised learning and greedy layer wise training approach are expected for better dimensionality reduction in microarray datasets compared with other methods.

\vspace*{0.2cm}

\noindent Keywords: \\ 
\noindent \textit{Microarray, ekspresi gen, Algoritma Clustering, feature selection, deep learning, unsupervised learning.}

\newpage