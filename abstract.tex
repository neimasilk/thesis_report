%
% Halaman Abstract
%
% @author  Andreas Febrian
% @version 1.00
%

\chapter*{ABSTRACT}

\vspace*{0.2cm}

\noindent \begin{tabular}{l l p{11.0cm}}
	Name&: & \penulis \\
	Program&: & \program \\
	Title&: & \judulInggris \\
\end{tabular} \\ 

\vspace*{0.5cm}

\noindent 
Gene expression data acquired by microarray experiments has a characteristic that the number of samples usually small but the number of features are very large. Deep Belief Network (DBN) is part of deep learning algorithms which apply unsupervised learning with greedy layer wise training technique. DBN can be used to analyse gene expression data.  Feature selection algorithm used by this study is based on multi-step weight based ranking extracted from DBN model to search biomarker from gene expression profile. This algorithm is applied for lung’s adenocarcinoma microarray dataset. DBNs can be viewed as a deep composition of simple, unsupervised networks of restricted Boltzmann machines (RBMs). This technique can solve the problem of searching biomarker extracted from microarray dataset. We evaluate the biomarker found by this method by using the biomarker as an input data to a supervised machine learning method using multi layers perceptron (MLP). We evaluate this MLP by analyzing the accuracy of classification problem from cancerous and healthy microarray’s patients data. As a confirmation, we conduct literature study about biomarker’s genes found by this methods.
\vspace*{0.2cm}

\noindent Keywords: \\ 
\noindent \textit{Microarray, gene expression, Feature Selection Algoritm, deep learning, deep belief networks, restricted boltzmann machine, unsupervised learning, greedy layer-wise training, biomarker.}

\newpage