%
% Halaman Abstract
%
% @author  Andreas Febrian
% @version 1.00
%

\chapter*{ABSTRACT}

\vspace*{0.2cm}

\noindent \begin{tabular}{l l p{11.0cm}}
	Name&: & \penulis \\
	Program&: & \program \\
	Title&: & \judulInggris \\
\end{tabular} \\ 

\vspace*{0.5cm}

\noindent 
Microarray technology has made possible the profiling of gene expressions of the entire genome in a single hybridization experiment. Since microarray data acquire tens of thousands of gene expression values simultaneously. However, the number of sample usually small. Deep learning architecture used in this experiment is Deep Belief Network (DBN). DBNs can be viewed as a composition of simple, unsupervised networks of restricted Boltzmann machines (RBMs). Feature selection algorithm used by this study is based on multi-step weight ranking extracted from DBN model. This algorithm is applied for lung's adenocarcinoma microarray dataset to extract informative biomarker's genes. This technique can solve the problem of feature selection extracted from microarray dataset. We evaluate the biomarker found by this method by using the biomarker as an input data to a supervised machine learning method using multi layers perceptron (MLP). By analyzing the accuracy of classification problem from cancerous and healthy microarray's patients data. As a confirmation, we conduct literature study about biomarker's genes found by this methods. 

\vspace*{0.2cm}

\noindent Keywords: \\ 
\noindent \textit{Microarray, gene expression, Feature Selection Algoritm, deep learning, deep belief networks, restricted boltzmann machine, unsupervised learning, greedy layer-wise training, biomarker.}

\newpage