%-----------------------------------------------------------------------------%
\chapter{\babTiga}
%-----------------------------------------------------------------------------%
\todo{tambahkan kata-kata pengantar bab 3 disini}
Penelitian ini dibagi menjadi tiga tahap: (1) Mendapatkan data microarray dan pengolahan awal; (2) Perancangan algoritma; (3) Testing dan kemudian dilanjutkan dengan evaluasi.


%-----------------------------------------------------------------------------%
\section{Gambaran Umum Penelitian}
%-----------------------------------------------------------------------------%
\todo{bikin dan tambahkan overview penelitian terbaru}
Gambar overview penelitian

%-----------------------------------------------------------------------------%
\section{Pengumpulan Data dan Pengolahan Awal}

%-----------------------------------------------------------------------------%
Data microarray tersedia secara bebas di geo [http://www.ncbi.nlm.nih.gov/geo/], dan dapat diunduh, untuk digunakan sebagai data penelitian. Kemudian dilakukan normalisasi standar yang sering di pakai pada data microarray, proses normalisasi ada banyak metode, dan akan digunakan satu metode standar untuk pengolahan awal microarray agar mendapatkan data konsisten dan dapat dibandingkan. Proses pengolahan awal dan normalisasi digunakan tools standar dan tersedia bebas yaitu R-Bioconductor.
\todo{ tambahkan bagan pengumpulan data dan pengolahan awal}


%-----------------------------------------------------------------------------%
\section{Perancangan Algoritma}

%-----------------------------------------------------------------------------%
\todo {bawah masih amburadul tambahkan bagan2 algoritma pada presentasi}
Algoritma deep learning yang dipakai adalah DBN dengan teknik ranking multi step bobot adalah modifikasi dari algoritma seleksi fitur Sevade (Shevade, 2006)
Ada tiga algoritma clustering yang diusulkan untuk diteliti, yaitu : (1) Cluster Affinity Search Technique (CAST) ; (2) K-Means Clustering ; (3) Hierarchical Clustering.
Metode evaluasi yang dipakai adalah :  Adjusted Rand Index  yang digunakan untuk mengevaluasi baik tidaknya hasil clustering.
3.4 Melakukan Testing Arsitektur Deep Learning
Hasil dari unsupervised learning yang dilakukan oleh deep learning, harus diuji dahulu dengan dengan data testing, apakah error rekonstruksinya masih baik atau rekonstruksi tersebut lebih jelek. Setelah dilakukan perankingan biomarker, diperlukan pengujian apakah apakah seleksi fitur tersebut menggambarkan hasil yang diinginkan, dengan membandingkan biomarker yang dihasilkan dengan literature.

%-----------------------------------------------------------------------------%
\section{Evaluasi Hasil Kombinasi Seleksi Fitur Secara Unsupervised dan Klasifikasi MLP Secara Supervised}

%-----------------------------------------------------------------------------%
\todo { pikirkan lagi judul section }



