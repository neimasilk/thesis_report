%---------------------------------------------------------------
\chapter{\kesimpulan}
%---------------------------------------------------------------
	

%---------------------------------------------------------------
\section{Kesimpulan}
%---------------------------------------------------------------
Penelitian ini menerapkan seleksi fitur perankingan multi-step pada \textit{arsitektur deep belief network (DBN)} untuk mencari \textit{biomarker} pada data microarray penyakit kanker paru-paru. Penerapannya menggunakan library Theano pada bahasa pemrograman Python. Kesimpulan yang dapat diambil dari penelitian ini adalah sebagai berikut:
\begin{enumerate}
\item Metodologi pencarian \textit{biomarker} secara \textit{unsupervised} dengan menggunakan teknik \textit{Deep Belief Network (DBN)} didapatkan model terbaik  dengan konfigurasi hidden unit 4 layer [7000, 10000, 5000, 1000] dengan epoch 1000 dan learning rate 0.01.
\item Algoritma perankingan gen secara multi-step yang diajukan pada thesis ini, bisa dilakukan untuk network DBN yang di training secara \textit{unsupervised} murni, dan menghasilkan hasil biomarker yang memiliki signifikansi yang tinggi.
\item Evaluasi yang dilakukan secara bertahap yaitu mulai dari dibandingkannya metode unsupervised dengan masalah klasifikasi \textit{supervised} dengan MLP menunjukkan peningkatan hasil klasifikasi yang signifikan. Dan \textit{biomarker} yang ditemukan, dibandingkan dengan literatur yaitu metode bonferroni menunjukkan bahwa gen yang ditemukan memiliki signifikansi yang tinggi.
\end{enumerate}




%---------------------------------------------------------------
\section{Saran}
%---------------------------------------------------------------

Karena keterbatasan waktu penelitian dan mesin yang digunakan, maka ada banyak hal yang bisa dilakukan untuk penelitian selanjutnya yaitu: 
\begin{enumerate}
\item Melakukan generalisasi, apakah metode ini cocok juga dilakukan untuk data \textit{microarray} pada penyakit-penyakit lainnya selain kanker paru-paru.
\item Karena metode ini menggunakan arsitektur \textit{deep learning} dengan jaringan DBN, apakah dengan melakukan pada network DBN yang lebih dalam (layer hidden dengan kedalaman lebih dari 4 layer) bisa meningkatkan keakuratan pendeteksian \textit{biomarker}. Dikarenakan terbatasnya memory komputer, maka hal ini belum memungkinkan untuk dilakukan.
\item Diterapkan arsitektur deep learning yang lainnya misalnya \textit{stacked autoencoder, denoising autoencoder, convolutional neural-network}, dan atau arsitektur-arsitektur deep learning yang baru.
\end{enumerate}

