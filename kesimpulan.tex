%---------------------------------------------------------------
\chapter{\kesimpulan}
%---------------------------------------------------------------
	

%---------------------------------------------------------------
\section{Kesimpulan}
%---------------------------------------------------------------
Metodologi pencarian biomarker secara \textit{unsupervised} dengan menggunakan teknik Deep Believe Network (DBN) memiliki kelebihan yaitu tidak diperlukannya pengetahuan kita tentang gen secara lengkap, dan bisa dilakukan generalisasi pada penyakit-penyakit lainnya.\\
Algoritma perankingan gen secara multi-step pada jaringan DBN yang merupakan modifikasi dari metode sebelumnya yang hanya bisa dilakukan pada teknik \textit{logistic regression} sekarang bisa dilakukan untuk network DBN yang di training secara unsupervised murni.\\
Evaluasi yang dilakukan secara bertahap yaitu mulai dari dibandingkannya metode unsupervised dengan masalah klasifikasi supervised dengan MLP menunjukkan peningkatan hasil klasifikasi yang signifikan. Dan biomarker yang ditemukan, dibandingkan dengan literatur yaitu metode bonferroni menunjukkan bahwa gen yang ditemukan memiliki signifikansi yang tinggi.



%---------------------------------------------------------------
\section{Saran}
%---------------------------------------------------------------

Karena keterbatasan waktu penelitian dan mesin yang digunakan, maka ada banyak hal yang bisa dilakukan untuk penelitian selanjutnya, yaitu melakukan generalisasi, apakah metode ini cocok juga dilakukan untuk penyakit-penyakit lainnya. \\
Karena metode ini menggunakan arsitektur deep learning yang memiliki jaringan yang dalam, apakah dengan melakukan pada network DBN yang lebih dalam bisa meningkatkan keakuratan pendeteksian biomarker. Dikarenakan terbatasnya komputer, makan hal ini belum memungkinkan untuk dilakukan.