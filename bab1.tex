%-----------------------------------------------------------------------------%
\chapter{\babSatu}
%-----------------------------------------------------------------------------%

%-----------------------------------------------------------------------------%
\section{Latar Belakang}
%-----------------------------------------------------------------------------%
Salah satu kegunaan teknologi microarray adalah untuk menemukan dan mendeteksi penyakit yang berhubungan dengan masalah genetis yang salah satunya adalah penyakit kanker. Percobaan microarray dilakukan untuk mendapatkan profil gen yang diambil dari sel kanker pasien. Profil gen yang didapatkan dalam percobaan ini adalah dalam bentuk ekspresi gen. Data ekspresi gen pada percobaan \textit{microarray} memiliki ciri khas yaitu dimensi fitur gen yang jauh lebih besar dibandingkan dengan sampel pasien. Hal ini dikarenakan oleh mahalnya percobaan dan terbatasnya pasien yang diteliti. Masalah tersebut menyebabkan penerapan teknik-teknik pendeteksian penyakit dengan menggunakan data ekspresi gen tersebut memiliki tantangan tersendiri. Salah satu teknik yang diterapkan untuk tujuan pendeteksian tersebut adalah menggunakan teknik \textit{machine learning}. Yaitu teknik pembelajaran komputer untuk pengenalan pola. Dengan teknik ini, pola penyakit yang diinginkan untuk dideteksi didapat dari pengolahan data profil ekspresi gen pasien. Menurut penelitian \cite{yoon2006building} dan \cite{bandyopadhyay2014survey} tidak semua gen yang didapatkan dalam percobaan microarray tersebut adalah gen yang informatif, bahkan jumlah ekspresi gen yang informatif untuk kasus yang diinginkan misalnya untuk pengenalan sel kanker, sangat sedikit dibandingkan dengan keseluruhan ekspresi gen yang didapatkan dalam sebuah eksperimen \citep{bandyopadhyay2014survey}. Data ekspresi gen yang tidak informatif tersebut dapat mengurangi performa  proses pengenalan pola secara signifikan pada teknik \textit{machine learning} yang diterapkan. Akan tetapi, beberapa gen yang informatif berpengaruh secara signifikan terhadap pengenalan pola tersebut. Sebagai contoh, untuk mendiagnosa kanker paru-paru, hanya dibutuhkan sekitar 50 gen saja dari 22 ribu gen yang didapatkan dalam percobaan. Gen-gen yang paling informatif ini disebut dengan Biomarker \citep{belinsky2004gene}. Sehingga, hanya dengan menggunakan data biomarker yang ditemukan, bisa digunakan untuk mengenali penyakit yang diderita oleh pasien.\\

Pendekatan unsupervised learning, sering dipilih untuk kasus pengenalan pola yang polanya tidak kita ketahui secara lengkap. Seperti kasus ekspresi gen pada percobaan microarray. Dikarenakan pengetahuan manusia tentang gen sampai saat ini masih terbatas, yaitu masih 26\% dari keseluruhan gen yang belum diketahui kegunaannya \citep{haggstrom2014diagram}. Oleh karena itu pendekatan teknik machine learning secara unsupervised sering dilakukan untuk analisa pengenalan pola data microarray. Pada penelitian ini, akan dilakukan seleksi fitur terhadap data microarray secara unsupervised learning dengan menggunakan teknik deep learning Deep Believe Network (DBN). Untuk kasus pengenalan pola pendeteksian penyakit kanker paru-paru pada sample pasien sakit dan normal. \\

Pada penelitian ini, akan dibangun sebuah teknik seleksi fitur dengan menggunakan metode yang dimodifikasi dari algoritma seleksi fitur untuk logistic regression yang dilakukan oleh \cite{shevade2003simple}. Dikarenakan algoritma seleksi fitur menggunakan logistic regression merupakan bagian dari supervised learning, maka dianggap kurang cocok untuk data microarray yang fungsinya belum diketahui secara lengkap dan memiliki karakteristik yang kompleks. Logistic regression juga memiliki masalah dalam  mengeliminasi fitur, dikarenakan koefisien bobot ditempatkan disetiap fitur. Oleh karena itu, disini akan diajukan arsitektur Deep Learning. Arsitektur deep learning yang akan digunakan pada penelitian ini adalah arsitektur Deep Belief Network (DBN). DBN merupakan jaringan Restrictive Boltzmann Machine (RBM) yang disusun secara bertingkat. Dimulai dengan memberikan bobot random diantara dua network, yang dapat dilatih dengan cara meminimalkan perbedaan antara data asli dengan data rekonstruksinya. Gradien didapatkan dengan chain rule untuk melakukan penurunan error dengan teknik Contrastive Divergence (CD). Untuk dicari bobot (W) dengan maximum likelihood learning  secara greedy per layernya (greedy layer wise training) \cite{hinton2006reducing}. Pada penelitian ini, dalam mencari perangkingan bobotnya, menggunakan modifikasi dari cara yang digunakan oleh Shevade \cite{shevade2003simple} dalam teknik seleksi fitur berbasis weight menggunakan Sparse logistic regression (Shevade, 2003). Teknik perangkingan weight ini akan dimodifikasi dan digunakan untuk  meranking fiturnya secara multi step yang diterapkan pada DBN. \\

Tahap selanjutnya fitur yang telah didapatkan pada tahap seleksi fitur, akan digunakan sebagai data untuk penerapan evaluasi apakah biomarker yang ditemukan dengan perankingan tersebut memperbaiki hasil klasifikasi. Untuk mengevaluasi dan menganalisa seberapa baik hasil gen biomarker yang dipilih tersebut informatif dan tidak, dilakukan perbandingan dengan literatur. \\






%-----------------------------------------------------------------------------%
\section{Perumusan Masalah}
%-----------------------------------------------------------------------------%
Dikarenakan karakteristik sedikitnya sampel dan besarnya fitur. Serta tidak lengkapnya informasi kita terhadap gen. Apakah pendekatan unsupervised pada deep learning untuk mencari biomarker dengan perankingan bobot secara multi step cocok dipakai pada data microarray?


%-----------------------------------------------------------------------------%
\subsection{Definisi Permasalahan}
%-----------------------------------------------------------------------------%
\todo{Tuliskan permasalahan yang ingin diselesaikan. Bisa juga
	berbentuk pertanyaan}


%-----------------------------------------------------------------------------%
\subsection{Batasan Permasalahan}
%-----------------------------------------------------------------------------%
\begin{itemize}
\item Dataset microarray
\item Data yang digunakan adalah dataset yang sudah dilakukan preprocessing standar dan sudah dinormalisasi.
\end{itemize}

%-----------------------------------------------------------------------------%
\section{Tujuan}
%-----------------------------------------------------------------------------%
Penelitian ini bertujuan untuk membangun metodologi dalam pencarian Biomarker Gen yang paling penting untuk percobaan microarray. Dengan menghitung bobot ranking gen secara multi step.

%-----------------------------------------------------------------------------%
\section{Posisi Penelitian}
%-----------------------------------------------------------------------------%
\todo{Posisi penelitian Anda jika dilihat secara bersamaan dengan 
	peneliti-peneliti lainnya. Akan lebih baik lagi jika ikut menyertakan 
	diagram yang menjelaskan hubungan dan keterkaitan antar 
	penelitian-penelitian sebelumnya}




%-----------------------------------------------------------------------------%
\section{Manfaat Penelitian}
%-----------------------------------------------------------------------------%
Mendapatkan framework cara perankingan data ekspresi gen menggunakan arsitektur deep learning. Sehingga membantu dalam pencarian biomarker pada penelitian data microarray.

%-----------------------------------------------------------------------------%
\section{Sistematika Penulisan}
%-----------------------------------------------------------------------------%
Sistematika penulisan laporan adalah sebagai berikut:
\begin{itemize}
	\item Bab 1 \babSatu \\
	Berisi gambaran permasalahan dan metodologi apa yang akan diterapakan
	\item Bab 2 \babDua \\
	Landasan teori dipakainya metodologi yang akan diterapkan dalam eksperimen
	\item Bab 3 \babTiga \\
	Penjelasan detail metodologi yang akan diterapkan dalam penelitian
	\item Bab 4 \babEmpat \\
	Pembahasan hasil dari eksperimen
	\item Bab 5 \kesimpulan \\
\end{itemize}

\todo{Tambahkan penjelasan singkat mengenai isi masing-masing bab.}

