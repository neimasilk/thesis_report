%-----------------------------------------------------------------------------%
\chapter{\babSatu}
%-----------------------------------------------------------------------------%

%-----------------------------------------------------------------------------%
\section{Latar Belakang}
%-----------------------------------------------------------------------------%
Data ekspresi gen pada percobaan \textit{microarray} memiliki ciri khas yaitu dimensi fitur gen yang jauh lebih besar dibandingkan dengan sampel pasien. Masalah tersebut menyebabkan penerapan teknik-teknik pendeteksian penyakit dengan menggunakan data ekspresi gen  lebih sulit dan memiliki tantangan tersendiri. Tidak semua data ekspresi gen memiliki signifikansi dalam pendeteksian penyakit. Menurut penelitian \cite{yoon2006building} dan \cite{bandyopadhyay2014survey} tidak semua gen yang didapatkan dalam percobaan \textit{microarray} tersebut adalah gen yang informatif, bahkan jumlah ekspresi gen yang informatif untuk kasus yang diinginkan misalnya untuk pengenalan sel kanker, sangat sedikit dibandingkan dengan keseluruhan ekspresi gen yang didapatkan dalam sebuah percobaan \citep{bandyopadhyay2014survey}. Data ekspresi gen yang tidak informatif tersebut dapat mengurangi performa  proses pengenalan pola secara signifikan pada teknik \textit{machine learning} yang diterapkan. Akan tetapi, beberapa gen yang informatif berpengaruh secara signifikan terhadap pengenalan pola tersebut. Sebagai contoh, untuk mendiagnosa kanker paru-paru, hanya dibutuhkan sekitar 50 gen saja dari 22 ribu gen yang didapatkan dalam percobaan. Gen-gen yang paling informatif ini disebut dengan \textit{Biomarker} \citep{belinsky2004gene}. Sehingga, hanya dengan menggunakan data \textit{Biomarker} yang ditemukan saja, sudah dapat digunakan untuk mengenali penyakit yang diderita oleh pasien.\\

\todo {motivasi dan kendala pencarian biomarker, \\
tahu2 muncul machine learning}

Menemukan Biomarker adalah xxx.
Pada penelitian ini, akan dibangun sebuah teknik pencarian \textit{Biomarker} dengan metode seleksi fitur gen. Metode yang digunakan merupakan modifikasi dari algoritma seleksi fitur untuk \textit{logistic regression} yang dilakukan oleh \cite{shevade2003simple}. Akan tetapi, \textit{Logistic regression} memiliki masalah dalam  mengeliminasi fitur, dikarenakan koefisien bobot ditempatkan disetiap fitur. Oleh karena itu, pada penelitian ini akan diajukan teknik pencarian \textit{Biomarker} dengan perankingan bobot pada arsitektur \textit{Deep Learning}. Teknik perangkingan bobot ini akan dimodifikasi dan digunakan untuk  meranking fitur gennya secara berjenjang atau \textit{multi step}. \\

\todo {hubungan antara deep larning, bobot dan multistep ranking}

Arsitektur \textit{deep learning} yang akan digunakan adalah arsitektur \textit{Deep Belief Network (DBN)}. DBN merupakan jaringan \textit{Restrictive Boltzmann Machine (RBM)} yang disusun secara bertingkat. Dimulai dengan memberikan bobot random diantara dua network, yang dapat dilatih dengan cara meminimalkan perbedaan antara data asli dengan data rekonstruksinya. \textit{Gradien} didapatkan dengan \textit{chain rule} untuk melakukan penurunan error dengan teknik \textit{Contrastive Divergence (CD)}. Untuk dicari bobot (W) dengan \textit{maximum likelihood learning}  secara \textit{greedy} pada tiap layernya \citep{hinton2006reducing}. \\

Tahapan berikutnya, fitur \textit{Biomarker} yang telah didapatkan akan digunakan sebagai data untuk penerapan evaluasi apakah \textit{Biomarker} yang ditemukan dengan perankingan tersebut dapat memperbaiki hasil klasifikasi. Untuk mengevaluasi seberapa akurat gen biomarker yang dipilih, dilakukan perbandingan hasil eksperimen dengan literatur. \\






%-----------------------------------------------------------------------------%
\section{Perumusan Masalah}
%-----------------------------------------------------------------------------%
Dikarenakan karakteristik sedikitnya sampel dan besarnya fitur. Serta tidak lengkapnya informasi kita terhadap gen. Apakah pendekatan unsupervised pada deep learning untuk mencari \textit{Biomarker} dengan perankingan bobot secara \textit{multi step} cocok dipakai pada data microarray?


%-----------------------------------------------------------------------------%
\subsection{Definisi Permasalahan}
%-----------------------------------------------------------------------------%
\todo{Tuliskan permasalahan yang ingin diselesaikan. Bisa juga
	berbentuk pertanyaan}


%-----------------------------------------------------------------------------%
\subsection{Batasan Permasalahan}
%-----------------------------------------------------------------------------%

\todo {pikirkan lagi batasan permasalahan}
\begin{itemize}
\item Dataset microarray untuk kanker paru-paru yang tersedia secara bebas dengan kode GSE10072
\item Data yang digunakan adalah dataset yang sudah dilakukan preprocessing standar dan sudah dinormalisasi.
\end{itemize}

%-----------------------------------------------------------------------------%
\section{Tujuan}
%-----------------------------------------------------------------------------%
Penelitian ini bertujuan untuk membangun metodologi dalam pencarian \textit{Biomarker} Gen yang paling penting untuk percobaan microarray. Dengan menghitung bobot ranking gen secara multi step.

%-----------------------------------------------------------------------------%
\section{Posisi Penelitian}
%-----------------------------------------------------------------------------%
\todo{Posisi penelitian Anda jika dilihat secara bersamaan dengan 
	peneliti-peneliti lainnya. Akan lebih baik lagi jika ikut menyertakan 
	diagram yang menjelaskan hubungan dan keterkaitan antar 
	penelitian-penelitian sebelumnya}




%-----------------------------------------------------------------------------%
\section{Manfaat Penelitian}
%-----------------------------------------------------------------------------%
Mendapatkan framework untuk melakukan perankingan data ekspresi gen sehingga didapatkan gen yang paling informatif menggunakan arsitektur deep learning agar membantu dalam pencarian biomarker pada penelitian data microarray.

%-----------------------------------------------------------------------------%
\section{Sistematika Penulisan}
%-----------------------------------------------------------------------------%
Sistematika penulisan laporan adalah sebagai berikut:
\begin{itemize}
	\item Bab 1 \babSatu \\
	Berisi gambaran permasalahan dan metodologi apa yang akan diterapakan
	\item Bab 2 \babDua \\
	Landasan teori dipakainya metodologi yang akan diterapkan dalam eksperimen
	\item Bab 3 \babTiga \\
	Penjelasan detail metodologi yang akan diterapkan dalam penelitian
	\item Bab 4 \babEmpat \\
	Pembahasan hasil dari eksperimen
	\item Bab 5 \kesimpulan \\
\end{itemize}

\todo{Tambahkan penjelasan singkat mengenai isi masing-masing bab.}

