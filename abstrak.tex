%
% Halaman Abstrak
%
% @author  Andreas Febrian
% @version 1.00
%

\chapter*{Abstrak}

\vspace*{0.2cm}

\noindent \begin{tabular}{l l p{10cm}}
	Nama&: & \penulis \\
	Program Studi&: & \program \\
	Judul&: & \judul \\
\end{tabular} \\ 

\vspace*{0.5cm}

\noindent 
Data ekspresi gen pada percobaan microarray memiliki ciri khas yaitu jumlah sampel yang sedikit dengan dimensi fitur yang sangat besar. Algoritma \textit{Deep Believe Network (DBN)}  adalah bagian dari algoritma \textit{deep learning} yang menerapkan teknik \textit{unsupervised learning} secara \textit{greedy layer wise training}. DBN ini dapat digunakan untuk membantu menganalisa data ekspresi gen. Algoritma seleksi fitur yang berbasis pada perankingan bobot secara multi-step pada penelitian ini digunakan untuk mendapatkan fitur gen \textit{biomarker}, yaitu profil gen yang paling informatif dengan melakukan perankingan berdasarkan bobot jaringan \textit{deep believe network} (DBN). Algoritma ini digunakan untuk  memilih fitur gen dari suatu percobaan microarray \textit{lung adenocarcinoma} (kanker paru-paru). \textit{Deep Believe Network} (DBN) adalah \textit{Restricted Boltzmann Machine} (RBM) yang dirangkai menjadi jaringan yang  dijajarkan untuk membentuk jaringan yang lebih dalam. Seleksi fitur gen, berdasarkan ranking bobot yang dihasilkan oleh algoritma ini terbukti dapat digunakan untuk pencarian \textit{Biomarker}. Hal ini dibuktikan dengan melakukan evaluasi bahwa hanya dengan menggunakan \textit{biomarker} yang didapatkan sebagai data pada teknik \textit{machine learning} umum yaitu \textit{multi layers perceptron}, sudah bisa melakukan klasifikasi pasien sehat atau pasien sakit. Untuk melakukan konfirmasi bahwa gen \textit{biomarker} tersebut adalah merupakan \textit{biomarker} dari penyakit kanker, maka dilakukan perbandingan dengan hasil dari studi literatur.

\vspace*{0.2cm}

\noindent Kata Kunci: \\ 
\noindent 
\textit{Microarray, ekspresi gen, Algoritma Seleksi fitur, multi-step ranking, deep believe network, restricted boltzmann machine, feature selection, deep learning, unsupervised learning, biomarker.}

\newpage