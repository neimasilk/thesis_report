%
% Halaman Abstrak
%
% @author  Andreas Febrian
% @version 1.00
%

\chapter*{Abstrak}

\vspace*{0.2cm}

\noindent \begin{tabular}{l l p{10cm}}
	Nama&: & \penulis \\
	Program Studi&: & \program \\
	Judul&: & \judul \\
\end{tabular} \\ 

\vspace*{0.5cm}

\noindent 
Data ekspresi gen pada percobaan microarray memiliki ciri khas yaitu jumlah sampel yang sedikit dengan dimensi fitur yang sangat banyak. Algoritma \textit{clustering}  dan algoritma \textit{deep learning} merupakan dua algoritma \textit{unsupervised learning}, yang bisa membantu menganalisa data ekspresi gen. Algoritma pemilihan fitur digunakan untuk mendapatkan fitur gen yang paling penting. Dan kemudian akan digunakan algoritma \textit{clustering} untuk mendapatkan struktur cluster dari data ekspresi gen. Pemilihan fitur yang paling informatif dari suatu kasus percobaan microarray, merupakan  masalah yang ada pada pemrosesan data ekspresi gen. Sehingga diperlukan eksplorasi lebih lanjut untuk pemilihan fiturnya. Seleksi fitur gen, berdasarkan ranking bobot yang dihasilkan oleh \textit{deep learning}, diharapkan dapat memecahkan masalah seleksi fitur tersebut. Sedangkan metode clustering yang dipakai adalah: Cluster Affinity Search Technique (CAST) , K-Means Clustering dan Hierarchical Clustering. Deep learning adalah metode pembelajaran mesin yang merupakan bagian dari algoritma neural network, metode deep learning yang sering dipakai adalah arsitektur deep believe network (DBN). Pendekatan unsupervised learning pada deep learning dan clustering, diharapkan dapat digunakan untuk membantu peneliti dalam menganalisa data ekspresi gen-nya.

\vspace*{0.2cm}

\noindent Kata Kunci: \\ 
\noindent 
\textit{Microarray, ekspresi gen, Algoritma Clustering, feature selection, deep learning, unsupervised learning.}

\newpage